\documentclass[12pt]{article}

\usepackage{amsmath}
\usepackage{amssymb}
\usepackage{amsthm}
\usepackage{mathtools}
\usepackage[margin=1.25in]{geometry}

\newcommand{\RR}{\mathbb{R}}
\newcommand{\NN}{\mathbb{N}}
\newcommand{\DD}{\mathcal{D}}
\newcommand{\inv}{{}^{-1}}
\newcommand{\one}[1]{\mathbb{1}_{#1}}
\renewcommand{\hat}{\widehat}
\DeclarePairedDelimiter{\ip}{\langle}{\rangle}
\DeclarePairedDelimiter{\norm}{\|}{\|}
\DeclarePairedDelimiter{\abs}{|}{|}
\DeclarePairedDelimiter{\set}{\{}{\}}

\title{%
  The Sortino ratio and distributions
}

\begin{document}

\maketitle

\section{Introduction}

The \textbf{Sortino ratio} is defined by
\begin{equation*}
  S = \frac{R-T}{\textit{DR}},
\end{equation*}
where $R$ is the rate of return of an asset, $T$ is a threshold rate of return,
and $\textit{DR}$ is the \textbf{downside variation} of the asset, defined by
\begin{equation*}
  \textit{DR} = \sqrt{
    \int_{-\infty}^{T} (T-r)^2 \cdot f(r) \ dr
  }.
\end{equation*}
Here, $f$ is the probability density function of the returns. The implicit
assumption is that $f$ is integrable, which is a problem for our purposes.

Instead, we only have the empirical cumulative distribution function. For the
sequence of observed returns $R_1, R_2, R_3, \dots$, the $n$th update of the
empirical \textsc{cdf} is given by
\begin{equation*}
  \hat{F}_n(x) = n\inv \sum_{k=1}^{n} H(x - R_k),
\end{equation*}
where
\begin{equation*}
  H(x) = 
  \begin{cases}
    0 & x < 0 \\
    1 & x \geq 0
  \end{cases}
\end{equation*}
is the standard Heaviside function. Unfortunately, $H$ is not differentiable, or
even continuous, so $\hat{f}_n = \hat{F}_n'$ does not exist.

\section{Distributions}

Let $\DD \subset \RR^\RR$ be a suitably-chosen closed subspace of real
functions. We will pin down exactly what constitutes the space once we
understand the parameters of our problem. A \textbf{distribution} is an element
of the dual space $\DD^*$ of linear functionals which are continuous with
respect to a certain topology. Again, this topology will be determined later.
For $G \in \DD^*$ and $g \in \DD$, we write $\ip{G, g}$ to denote the
application $G(g)$. Some common distributions include
\begin{itemize}
  \item evaluation functionals: for $r_0 \in \RR$ let
    $\delta_{r_0} \in \DD^*$ be defined by $\ip{\delta_{r_0}, g} = g(r_0)$.
  \item integration functionals: for a suitably chosen function $\varphi$, let
    $I_\varphi \in \DD^*$ be defined by $\ip{I_\varphi, g} = \int \varphi g$.
\end{itemize}
We can use integration by parts to motivate a special kind of
\textbf{distributional derivative}, but this will depend on the specification of
$\DD$.

Let's examine the downside variation again. To expose the radicand, consider the
term
\begin{equation*}
  \textit{DR}^2 = \int_{-\infty}^{T} (T-r)^2 \cdot f(r) \ dr.
\end{equation*}

The existence and integrability of the probability density function $f$ is the
problem we are trying to fix. We want to turn $f$ into a suitable distribution,
which means we will need to identify the correct space of functions $\DD$. The
only function we know we \emph{need} to be in $\DD$ is $r \mapsto (T-r)^2$,
which is a $C^{\infty}( (-\infty,  T])$ function. On the other hand, $r \mapsto
(T-r)^2$ doesn't have the right asymptotic properties on which to assign a
topology of uniform convergence directly. This can be resolved by introducing a
weight function $w(r) = (T-r)^{-2}+1$ and by defining $\DD$ to be the subspace of
$C^{\infty}((-\infty, T])$ given by
\begin{equation*}
  \DD = \set{
    g \in C^{\infty}( (-\infty, T]) : 
    \text{
      $\norm{g^{(n)}}_{w} < \infty$ for all $n \in \NN$%
    }
  },
\end{equation*}
where
\begin{equation*}
  \norm{g}_w = \sup_{r \in \RR} \set{((T-r)^{-2}+1) \abs{g(r)}}.
\end{equation*}
Clearly, $r \mapsto (T-r)^2$ is a member of $\DD$, and there is a natural choice
of topology with the seminorms $\set{\rho_n}_{n = 1}^{\infty}$ defined by
$\rho_n(g) = \norm{g^{(n)}}_w$. Thus $g_k \to g$ in $\DD$ if and only if
$w \cdot g_k^{(n)} \to w\cdot g^{(n)}$ uniformly for each $n$. This, in turn,
implies that $g_k^{(n)} \to g^{(n)}$ uniformly on compact sets for each $n$.

With $\DD$ specified, we now need to worry about $\DD^*$, the space of
continuous linear functionals on $\DD$. In order for a linear functional $D$
on $\DD$ to be continuous, we will need the sequence $\ip{D, g_k}$ to
converge to $\ip{D, g}$ for every converging sequence $g_k \to g$ in $\DD$.
For the theory of distributions to be of use to us, then, we need to verify that
the Heaviside functions can be viewed as elements in $\DD^*$ and confirm that
their distributional derivatives are also elements in $\DD^*$.

Define the Heaviside functional $H_{r_0}$ by
\begin{equation*}
  \ip{H_{r_0}, g} = H(T-r_0) \int_{r_0}^{T} g(r) \ dr.
\end{equation*}
The term $H(T-r_0)$ is 1 if $r_0 \leq T$ and $0$ otherwise. This ensures that
the Heaviside functional is appropriately $0$ when the center $r_0$ is to the
right of $T$. We first verify that $H_{r_0}$ is linear, then verify its
continuity. Indeed, 
\begin{align*}
  \ip{H_{r_0}, c_1 g_1 + c_2 g_2} &= H(T - r_0) \cdot \int_0^T c_1 g_1(r)
  + c_2 g_2(r) \ dr \\
  &= c_1 H(T - r_0) \int_{r_0}^{T} g_1(r) \ dr + 
     c_2 H(T - r_0) \int_{r_0}^{T} g_2(r) \ dr \\
  &= c_1 \ip{H_{r_0}, g_1} + c_2 \ip{H_{r_0}, g_2},
\end{align*}
establishing linearity. Moreover, if $g_{k} \to g$ in $\DD$, then $g_k \to g$
uniformly on $[r_0, T]$. Therefore, all but finitely many $g_k$ are bounded
above by $g+1$; hence, $\int_{r_0}^T g_k \to \int_{r_0}^T g$ by the dominated
convergence theorem. This implies that $H_{r_0}$ is a continuous function.
Therefore, $H \in \DD^*$.

Since $\DD^*$ is a vector space, and since $H_{r_0} \in \DD^*$ for each $r_0 \in
\RR$, it follows that every empirical \textsc{cdf}
\begin{equation*}
  \hat{F}_n = n\inv \sum_{k=1}^{n} H_{R_k}
\end{equation*}
is in $\DD^*$, being a linear combination of Heavisides. The
remaining question is whether the distributional derivative of $\hat{F}_n$
resides in $\DD^*$. It suffices to consider whether $H_{R_0}' \in \DD^*$. We
needs first agree to a suitable \emph{notion} of differentiation. Imagining a
successful application of integration by parts with suitable functions, we would
expect that
\begin{align*}
  \int_{-\infty}^T g(r) \cdot H'(r-r_0) \ dr 
  &= g(r) \cdot H(r-r_0)
  \bigg|_{-\infty}^{T} - \int_{-\infty}^{T} g'(r) \cdot H(r-r_0) \ dr \\
  &= g(T) \cdot H(T-r_0) - \int_{r_0}^T g'(r) \ dr \\
  &= g(T) \cdot H(T-r_0) - (g(T) - g(r_0)) \\
  &= g(T) \cdot [H(T-r_0) - 1] + g(r_0).
\end{align*}
For consistency, we \emph{define} $H_{r_0}'$ by
\begin{equation*}
  \ip{H_{r_0}', g} =  H(T-r_0)\cdot [H(T-r_0) - 1]\cdot g(T)  + H(T-r_0) \cdot
  g(r_0).
\end{equation*}
It is clear that $H_{r_0}'$ is linear and continuous, so $H_{r_0}' \in \DD^*$.
Therefore $\hat{F}_n' \in \DD^*$ for each $n \in \NN$.

\section{Conclusion}

We can now evaluate $\hat{F}_{n}'$. In particular, we have
\begin{align*}
  \ip{\hat{F}_n', g} &= n\inv \sum_{k=1}^{n} \ip{H_{R_k}', g} \\
  &= n\inv \sum_{k=1}^{n} \left[ 
    H(T-R_k)\cdot [H(T-R_k) - 1]\cdot g(T)  + H(T-R_k) \cdot
    g(R_k)
  \right] \\
  &= n\inv \cdot g(T) \sum_{k=1}^{n} H(T-R_k) \cdot \left[ 
    H(T-R_k) - 1 
  \right] 
  + n\inv \sum_{k=1}^{n} H(T-R_k) \cdot g(R_k).
\end{align*}
This is the evaluation on an arbitrary $g \in \DD$. For the specific function
$g(r) = (T-r)^2$ in the Sortino ratio, this becomes
\begin{equation*}
  \ip{\hat{F}_n', g} = n\inv \sum_{k=1}^{n} H(T-R_k) \cdot (T-R_k)^2
\end{equation*}
since $g(T) = 0$.  The $n$th estimate of the Sortino ratio is, therefore, given
by
\begin{equation*}
  \widehat{S}_n = \frac{R_n - T}{\hat{\textit{DR}}_n} = 
  \frac{R_n - T}{\sqrt{n\inv \sum_{k=1}^{n} H(T-R_k) \cdot (T-R_k)^2}}.
\end{equation*}
Interestingly, and perhaps coincidentally, the computation of $\ip{\hat{F}_n',
g}$ corresponds to the uniform Monte Carlo sampling estimate of the original
integral.

\end{document}
